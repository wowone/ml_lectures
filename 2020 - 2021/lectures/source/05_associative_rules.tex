\documentclass[9pt]{beamer}
\usepackage{styles/mypreamble}
%~~~~~~~~~~~~~~~~~~~~~~~~~~~~~~~~~~~~~~~~~~~~~~~~~~~~~~~~~~~~~~~~~~~~~~~~~~~~~~
\title{Алгоритмы машинного обучения}
\subtitle{Лекция 5. Ассоциативные правила}
\author{Владимир Кукушкин}
\institute{СПбГЭУ - 16.12.2020}
%~~~~~~~~~~~~~~~~~~~~~~~~~~~~~~~~~~~~~~~~~~~~~~~~~~~~~~~~~~~~~~~~~~~~~~~~~~~~~~

\begin{document}

\titlepage

\section{Постановка задачи}

\begin{frame}{Что это такое}
\begin{itemize}
    \item Пользователи, которые купили X также покупают Y.
    \item Предшественник рекомендательных систем. 
    \item Одна из классических прикладных задач. Магазины массово стали вкладываться в решения в 90х.
\end{itemize}
\end{frame}

\begin{frame}{Формализация задачи}
\begin{itemize}
    \item Данные состоят из бинарных колонок $Z_j$. Строка ассоциируется с одной покупкой в супермаркете (чеком), колонки -- с товарами. Всего в магазине $K$ различных товаров.
    $$z_{ij} = \begin{cases}
    1, \text{если в $i$-м чеке был куплен $j$-й товар};\\
    0, \text{иначе}
    \end{cases}.$$
    \item Хочется найти такие подмножества товаров $\mathcal{K}\subset \{1, \ldots, K\}$, чтобы вероятность их совместной покупки в одном чеке
    $$\text{Pr}\left[\underset{k\in \mathcal{K}}{\bigcap} (Z_k = 1)\right] = \text{Pr}\left[\underset{k\in \mathcal{K}}{\prod} Z_k = 1\right]$$
    была достаточно большой.
    \item Можно оценить эту вероятность на основе имеющихся данных:
    $$T(\mathcal{K)} = \widehat{\text{Pr}}\left[\underset{k\in \mathcal{K}}{\prod} Z_k = 1\right] = \frac{1}{N}\sum_{i=1}^N \underset{k\in \mathcal{K}}{\prod} z_{ik}.$$
    $T(\mathcal{K)}$ называют поддержкой (support, prevalence).
\end{itemize}
\end{frame}

\begin{frame}{Замечания}
\begin{itemize}
    \item Хотим найти все такие множества
    \begin{equation}\label{ass_rule_main}\{\mathcal{K}_l\;|\;T(\mathcal{K}_l) > t\}\end{equation}
    для наперёд заданного $t$.
    \item Брутфорс не сработает. (Почему?)
    \item Мощность $|\{\mathcal{K}_l\}|$ будет небольшой.
    \item Если $\mathcal{L} \subseteq \mathcal{K}$, то $T(\mathcal{L}) \geq T(\mathcal{K})$.
\end{itemize}
Все эти замечания наводят на мысль о The Apriori Alorithm (1995).
\end{frame}

\section{The Apriori Algorithm}

\begin{frame}{Описание алгоритма}
\begin{itemize}
    \item Сначала проверяем все подмножества товаров из одного элемента. Отсекаем те, для которых не выполнено $\eqref{ass_rule_main}$.
    \item Добавляем к выжившим подможествам из предыдущего пункта второй элемент. Снова проверяем, кто останется среди них.
    \item То есть чтобы проверить какое-то подмножество мощности $|\mathcal{K}| = m$, нужно перепробовать все подмножества этого множества из $m-1$ элемента.
\end{itemize}
\end{frame}

\begin{frame}{Ассоциативные правила}
\begin{itemize}
    \item В Apriori algorithm при переходе от $m-1$ к $m$, расширяя подмножество, мы оцениваем, как часто с данными $m-1$ товарами покупают $m$-й.
    \item Обощение. Если $\mathcal{K} = A\sqcup B$, то говорят, что задано ассоциативное правило $A \Rightarrow B$. $A$ -- условие (antecedent), $B$ -- следствие (consequent).
    \item Ассоциативные правила характеризуются двумя важными свойствами:
    \begin{itemize}
        \item \textbf{Достоверность} (confidence): 
        $$C(A \Rightarrow B) = \frac{T(A \Rightarrow B)}{T(A)}.$$
        На самом деле это просто $\text{Pr} (B|A)$.
        \item \textbf{Усиление} (lift):
        $$L(A \Rightarrow B) = \frac{C(A \Rightarrow B)}{T(B)}.$$
        Показывает, во сколько раз добавление $A$ к $B$ увеличивает вероятность покупки $B$. То есть это $\text{Pr}(A\text{ and }B) / \text{Pr}(A)\text{Pr}(B)$.
    \end{itemize}
    \item И теперь уже хотим искать правила, максимизирующие $C$ и/или $L$.
\end{itemize}
\end{frame}

\begin{frame}{Пример}
\begin{itemize}
    \item Проверим корзину $\mathcal{K} = \{\text{масло, сыр, хлеб}\}$ и рассмотрим ассоциативное правило $\{\text{масло, сыр} \Rightarrow \text{хлеб}\}$.
    \item Если саппорт этого правила равен 0.03, то это означает, что эти три товара появляются вместе в корзинах в 3\% чеков.
    \item Конфиденс в 82\% означает, что если покупаются масло с сыром, то в 82\% случаев будет вдобавок куплен и хлеб.
    \item Если при этом хлеб встречается в 43\% всех чеков, то тогда усиление покупки хлеба набором $\{\text{масло, сыр}\}$ составляет 1.95. 
\end{itemize}
\end{frame}



\begin{frame}[allowframebreaks]
    \frametitle{Литература}
    \bibliographystyle{unsrt}
    \nocite{esl}
    \nocite{arules_habr}
    \bibliography{references.bib}
\end{frame}

\end{document}